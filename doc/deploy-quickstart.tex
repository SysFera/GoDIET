\section{\godiet Quickstart}

Dans cette partie nous allons lancer une hi�rarchie \diet tr�s simple
avec \godiet. Pour simplifier encore les choses, \diet et \godiet
seront ex�cuter sur la m�me machine. Les pr�-requis pour r�aliser ce
sc�nario sont d'avoir install� \diet, OmniNames ainsi que le serveur
et le client \godiet.  

\subsection{Cr�ation des fichiers de configuration}

\begin{RQ}[Configuration files]
Les fichiers d�crits par la suite se situent dans le r�pertoire \texttt{examples/localhost} de l'archive \godiet client.
\end{RQ}

\begin{todo}[Check avec Philippe]
sauf que non... le fichier Infrastructure description est ici:
\texttt{project/target/test-classes/infrastructure/localhost-infrastructure.xml}
? 
\end{todo}

\begin{description}
\item[Infrastructure description] La premi�re �tape est de cr�er le fichier de description de l'infrastructure physique. Dans ce cas, un domaine,  nomm� DomainLocalhost contient votre machine nomm�e my-node. 
La seule contrainte de nommage est qu'il faut que l'identifiant soit
unique. \end{description}

%Copie du fichier infrastructure\localhost-infrastructure.xml des
%tests unitaires

\begin{center}
\begin{scriptsize}
\begin{boxedverbatim} 
<?xml version="1.0" encoding="UTF-8"?>
<godiet:infrastructure ...">
   <domain id="DomainLocalhost"/>
      <node id="mynode">
         <ssh id="lo" domain="DomainLocalhost" login="mylogin" server="127.0.0.1"/>
         <env>
            <var name="PATH"
                  value="/omniNamesInstallPath/bin:/dietInstallPath/bin:$PATH"/>
            <var name="LD_LIBRARY_PATH"
                  value="/omniNamesInstallPath/lib:/dietInstallPath/lib:"/>
         </env>
	<scratch dir="/tmp/scratch/"/>	
	</node>
</godiet:infrastructure>
\end{boxedverbatim}

\textbf{File: } project/target/test-classes/infrastructure/localhost-infrastructure.xml

\end{scriptsize}
\end{center}

Remplacer \texttt{mylogin} par votre login et les valeurs des variables
\texttt{PATH} et \texttt{LD\_LIBRARY\_PATH} par celles correspondant
aux r�pertoire d'installation sur votre machine.

\begin{description}
\item[Platform description] Le fichier ci-apr�s d�crit la plate-forme
  minimale pour exposer un service. Il est compos� d'un omniNames et
  d'un master Agent et un Sed fournissant un service de calcul
  matriciel. You can use the following file without any modification

\end{description}

\begin{center}
\begin{scriptsize}
\begin{boxedverbatim} 
<?xml version="1.0" encoding="UTF-8"?>
<godiet:dietPlatform xmlns:godiet="http://www.sysfera.com"
   xmlns:xsi="http://www.w3.org/2001/XMLSchema-instance"
   xsi:schemaLocation="http://www.sysfera.com ../Diet.xsd">

   <services>
      <omniNames id="omniNames" domain="DomainLocalhost">
         <config server="mynode" />
      </omniNames>
   </services>

   <hierarchy>
      <masterAgent id="MA1">
         <config server="mynode" />

         <sed id="matrix_service">
            <config server="mynode" />
            <file id="sed_config">
               <template name="sed_template.config" />
            </file>
            <binary name="dmat_manips_server">
               <commandLine>
                  <parameter string="T" />
               </commandLine>
            </binary>
         </sed>
      
      </masterAgent>
   </hierarchy>
</godiet:dietPlatform>
\end{boxedverbatim}

\textbf{File:} project/target/test-classes/diet/localhost-diet.xml
\end{scriptsize}
\end{center}

\subsection{D�marrer le serveur}

Deux fichiers pour l'instant server.properties et configuration.xml

Vous devez cr�er le fichier de configuration associ� au serveur. Il
d�crit le chemin ou se trouve les clefs \texttt{ssh} priv�es, le nom du noeud sur lequel il est d�marr�
You need to create a \godiet server configuration file. It contains information for remote connection like your \texttt{ssh} keys path. The server will look this file 
in \verb+${HOME}/.godiet/configuration.xml+ directory.

\begin{center}
\begin{scriptsize}
\begin{boxedverbatim} 
<?xml version="1.0" encoding="UTF-8"?>
<godiet:configuration xmlns:godiet="http://www.sysfera.com"
   xmlns:xsi="http://www.w3.org/2001/XMLSchema-instance"
   xsi:schemaLocation="http://www.sysfera.com ../Configuration.xsd"
   localNode="mynode">
   <localscratch dir="/tmp/sc_godiet" />
   <user>
      <ssh>
         <key path="myPrivateKeyPath" encrypted="false"/>
      </ssh>
   </user>
</godiet:configuration>
\end{boxedverbatim}

\textbf{File: } project/target/classes/configuration/configuration-local.xml
\end{scriptsize}
\end{center}

\begin{todo}[Check avec Philippe]
Le fichier en question a en fait des /home/phi en hard coded. 
\end{todo}

Remplacer myPrivateKeyPath par le chemin associ� � votre clef priv�e \texttt{ssh}.
Pour d�marrer le serveur, lancer le script run-server.sh

\subsection{D�marrer le client}





