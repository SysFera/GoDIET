\subsubsection{Quickstart}

Dans cette partie nous allons lancer une hierarchie \diet très simple avec \godiet. Pour simplifier encore les choses,\diet et \godiet seront executer sur la même machine. Les pré-requis pour réaliser ce scénario sont d'avoir installer \diet, OmniNames ainsi que le serveur et le client \godiet.
Les fichiers décrits par la suite se situent dans le répertoire examples/localhost de l'archive \godiet client.

\paragraph{Infratructure description}
La première étape est de créer le fichier de description de l'infrastructure physique. Dans ce cas, un domaine,  nommé DomainLocalhost contient votre machine nommée my-node. 
La seule contrainte de nommage est qu'il faut que l'identifiant soit unique.
s
%Copie du fichier infrastructure\localhost-infrastructure.xml des tests unitaires
\begin{verbatim}
<?xml version="1.0" encoding="UTF-8"?>
<godiet:infrastructure ...">
   <domain id="DomainLocalhost"/>
      <node id="mynode">
         <ssh id="lo" domain="DomainLocalhost" login="mylogin" server="127.0.0.1"/>
         <env>
            <var name="PATH"
                  value="/omniNamesInstallPath/bin:/dietInstallPath/bin:$PATH"/>
            <var name="LD_LIBRARY_PATH"
                  value="/omniNamesInstallPath/lib:/dietInstallPath/lib:"/>
         </env>
	<scratch dir="/tmp/scratch/"/>	
	</node>
</godiet:infrastructure>
\end{verbatim}

Remplacer mylogin par le votre et les valeur des variables PATH et LD\_LIBRARY\_PATH par celles correspondant aux répertoire d'installation sur votre machine.


\paragraph{Platform description}

Le fichier ci-après décrit la plate-forme minimale pour exposer un service. Il est composé d'un omniNames et d'un master Agent et un Sed fournissant un service de calcul matriciel. 

%Copie du fichier diet\localhost-diet des tests unitaires
\begin{verbatim}
<?xml version="1.0" encoding="UTF-8"?>
<godiet:dietPlatform xmlns:godiet="http://www.sysfera.com"
   xmlns:xsi="http://www.w3.org/2001/XMLSchema-instance"
   xsi:schemaLocation="http://www.sysfera.com ../Diet.xsd">

   <services>
      <omniNames id="omniNames" domain="DomainLocalhost">
         <config server="mynode" />
      </omniNames>
   </services>

   <hierarchy>
      <masterAgent id="MA1">
         <config server="mynode" />

         <sed id="matrix_service">
            <config server="mynode" />
            <file id="sed_config">
               <template name="sed_template.config" />
            </file>
            <binary name="dmat_manips_server">
               <commandLine>
                  <parameter string="T" />
               </commandLine>
            </binary>
         </sed>
      
      </masterAgent>
   </hierarchy>
</godiet:dietPlatform>
\end{verbatim}

Vous pouvez utiliser ce fichier tel quel.

\paragraph{Démarrer le serveur}

Deux fichiers pour l'instant server.properties et configuration.xml

Vous devez créer le fichier de configuration associé au serveur. Il décrit le chemin ou se trouve les clefs SSH, le nom du noeud sur lequel il est démarré
You need to create a \godiet server configuration file. It contains information for remote connection like your SSH keys path. The server will look this file 
in \verb+${HOME}/.godiet/configuration.xml+~\ref{GODIETConfiguration} directory.


\begin{verbatim}
<?xml version="1.0" encoding="UTF-8"?>
<godiet:configuration xmlns:godiet="http://www.sysfera.com"
   xmlns:xsi="http://www.w3.org/2001/XMLSchema-instance"
   xsi:schemaLocation="http://www.sysfera.com ../Configuration.xsd"
   localNode="mynode">
   <localscratch dir="/tmp/sc_godiet" />
   <user>
      <ssh>
         <key path="myPrivateKeyPath" encrypted="false"/>
      </ssh>
   </user>
</godiet:configuration>
\end{verbatim}

Remplacé myPrivateKeyPaht par le chemin associé à votre clef SSH.
Pour démarrer le serveur, lancé le script run-server.sh

\paragraph{Démarrer le client}





